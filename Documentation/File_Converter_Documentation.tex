%%%%%%%%%%%%%%%%%%%%%%%%%%%%%%%%%%%%%%%%%
% Sullivan Business Report
% LaTeX Template
% Version 1.0 (May 5, 2022)
%
% This template originates from:
% https://www.LaTeXTemplates.com
%
% Author:
% Vel (vel@latextemplates.com)
%
% License:
% CC BY-NC-SA 4.0 (https://creativecommons.org/licenses/by-nc-sa/4.0/)
%
%%%%%%%%%%%%%%%%%%%%%%%%%%%%%%%%%%%%%%%%%

%----------------------------------------------------------------------------------------
%	CLASS, PACKAGES AND OTHER DOCUMENT CONFIGURATIONS
%----------------------------------------------------------------------------------------

\documentclass[
	a4paper, % Paper size, use either a4paper or letterpaper
	12pt, % Default font size, the template is designed to look good at 12pt so it's best not to change this
	%unnumberedsections, % Uncomment for no section numbering
]{CSSullivanBusinessReport}

\usepackage{biblatex}
\addbibresource{Gangwar2023Feb.bib} % BibLaTeX bibliography file
\usepackage{listing}


%----------------------------------------------------------------------------------------
%	REPORT INFORMATION
%----------------------------------------------------------------------------------------

\reporttitle{File Converter Documentation} % The report title to appear on the title page and page headers, do not create manual new lines here as this will carry over to page headers


\reportauthors{\textbf{Marquis Skinner}\\GitHub: \url{https://github.com/MaybeMarq}} % Report authors/group/department, include new lines if needed

\reportdate{\today} % Report date, include new lines for additional information if needed


%----------------------------------------------------------------------------------------

\begin{document}

%----------------------------------------------------------------------------------------
%	TITLE PAGE
%----------------------------------------------------------------------------------------

\thispagestyle{empty} % Suppress headers and footers on this page

\begin{fullwidth} % Use the whole page width
	\vspace*{-0.075\textheight} % Pull logo into the top margin
	

	\vspace{0.15\textheight} % Vertical whitespace

	\parbox{0.9\fulltextwidth}{\fontsize{50pt}{52pt}\selectfont\raggedright\textbf{\reporttitle}\par} % Report title, intentionally at less than full width for nice wrapping. Adjust the width of the \parbox and the font size as needed for your title to look good.
	
	\vspace{0.03\textheight} % Vertical whitespace
	
	{\LARGE\textit{\textbf{\reportsubtitle}}\par} % Subtitle
	
	\vfill % Vertical whitespace
	
	{\Large\reportauthors\par} % Report authors, group or department
	
	\vfill\vfill\vfill % Vertical whitespace
	
	{\large\reportdate\par} % Report date
\end{fullwidth}

\newpage

%----------------------------------------------------------------------------------------
%	DISCLAIMER/COPYRIGHT PAGE
%----------------------------------------------------------------------------------------

\thispagestyle{empty} % Suppress headers and footers on this page



%----------------------------------------------------------------------------------------
%	TABLE OF CONTENTS
%----------------------------------------------------------------------------------------

\begin{twothirdswidth} % Content in this environment to be at two-thirds of the whole page width
	\tableofcontents % Output the table of contents, automatically generated from the section commands used in the document
\end{twothirdswidth}

\newpage

%----------------------------------------------------------------------------------------
%	SECTIONS
%----------------------------------------------------------------------------------------
\begin{fullwidth}
\section{What is File Converter?}
Tired of using sketchy, poorly working websites to convert your files? File Convertor is a simple command-line python script used to convert almost any image file into a JPEG or a .docx document into a .pdf document. It currently supports Windows and Linux. 
\section{Installing Python and Required Libraries} % Top level section

Python is the programming language used for this script. To run it, you  will need Python installed on your system. The process to install Python depends on your operating system. The details to install Python on the currently supported operating systems (Windows and Linux) are below.

\subsection{Installing Python on Windows}
Installing Python on Windows is a bit more involved than Linux. We are following the steps provided by Digital Ocean \cite{Gangwar2023Feb}.\\ 

% Step 1-------------------------------------------------------------------------------------------------------------------
\textbf{\underline{Step 1:  Downloading the Python Installer}}\\
\begin{enumerate}
\item Head to the \href{https://www.python.org/downloads/windows/}{\underline{official Python download page for Windows. }}
\item Click the link for the latest version and download the executable (.exe) file.
\end{enumerate}

%Step 2 ---------------------------------------------------------------------------------------------------------------
\textbf{\underline{Step 2:  Run the Installer}}\\
\begin{enumerate}
	\item Double click the executable 
	\item Select the \textbf{Install launcher for all users} checkbox. This enables all users of the computer to use the Python Launcher Application.
	\item Select the \textbf{Add python.exe to PATH} checkbox. This enables users to launch Python from the command line (very important).
	\item If this is your first time using Python and you are just getting started, we recommend clicking the \textbf{Install Now} option and skip to Step 4. To install other optional or advanced features, click \textbf{Customize Installation} and continue.
	\item If you want to look at the optional features you can, if not click \textbf{Next}.
	\item Make sure that the \textbf{Add Python to environment variables} option is checked. Choose the other items that suit you . 
	\item Click \textbf{Install}. A confirmation message will display when the installation is complete.
\end{enumerate}

%Step 3 ---------------------------------------------------------------------------------------------------------------
\textbf{\underline{Step 3:  Verify the Python Installation}}\\

\begin{enumerate}
	\item Go to \textbf{Start} and search for \textit{cmd}. Open \textbf{Command Prompt}.
	\item Enter the following command in the window. A version number should be output. If not, try the installation process again or visit the website included for more information.
	\begin{lstlisting}[language=bash]
		 python --version
	\end{lstlisting}
\end{enumerate}

\subsection{Installing Python on Linux} 
The installation of Python on Linux somewhat varies depending on the distro you are running. We have information regarding installation on Ubuntu (and related distros) and Fedora. For more niche distros, refer to your distro's wiki page for installation instructions (likely to be similar).
\subsubsection{Installing on Ubuntu:}
Installing on Ubuntu and related distros required the following command in the terminal:
\begin{lstlisting}[language=bash]
	$ sudo apt-get update
	$ sudo apt-get install python3
\end{lstlisting}

\subsubsection{Installing on Fedora:}
Installing on Fedora requires the following command in the terminal:
\begin{lstlisting}[language=bash]
	$ sudo dnf install python3
\end{lstlisting}

\subsection{Installing Libraries}
There are few external libraries this script relies on. They are listed below:
\begin{enumerate}
	\item \textbf{PIL} - Image manipulation library
	\item \textbf{docx2pdf} - Allows for the conversion from .docx to .pdf
	\item	\textbf{pillow-heif} - Additional library for manipulating .heif images (default file type for images taken on an iPhone)
\end{enumerate}

The installation process is pretty straightforward. Open up a terminal or command prompt and run these commands.

\begin{lstlisting}[language=bash]
	$ pip install Pillow
	$ pip install docx2pdf
	$ pip install pillow-heif
\end{lstlisting}

After running these commands, the supporting libraries should be installed on your system.

\section{Downloading File Converter}
Currently, there is no working executable file that allows plug and play. You can download the script and supporting images from the \href{https://github.com/MaybeMarq/File-Converter}{\underline{GitHub repository}}. You will also need to make sure the external libraries this script depends on can be accessed by the script. A simpler solution to this is being worked on but is currently not ready. 

\section{Running File Converter}
In your terminal or in your IDE of choice, go to the directory in which the script is stored. You can do this by using the 'cd' command followed by the name of the subsequent directories. When you are in the correct directory, run this command:

\begin{lstlisting}[language=bash]
	$ python3 FileConverter.py
	\end{lstlisting}
	
\section{Contributing}
This project was originally intended to be a simple tool for me to use. However, I thought it would be best to share my tool with anyone who wants to use it and I would love to see it continue to grow. If you would like to contribute whether that be in the form of  code, money, words of inspiration, or criticisms, I would wholeheartedly appreciate it. Please feel free to reach out to me. My contact information is  provided on my \href{https://github.com/MaybeMarq}{\underline{GitHub profile}}.

\newpage

\end{fullwidth}




%----------------------------------------------------------------------------------------

\printbibliography
\end{document}


